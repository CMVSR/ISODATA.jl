%%%%%%%%%%%%%%%%%%%%%%%%%%%%%%%%%%%%%%%%%
% Journal Article
% LaTeX Template
% Version 1.4 (15/5/16)
%
% This template has been downloaded from:
% http://www.LaTeXTemplates.com
%
% Original author:
% Frits Wenneker (http://www.howtotex.com) with extensive modifications by
% Vel (vel@LaTeXTemplates.com)
%
% License:
% CC BY-NC-SA 3.0 (http://creativecommons.org/licenses/by-nc-sa/3.0/)
%
%%%%%%%%%%%%%%%%%%%%%%%%%%%%%%%%%%%%%%%%%

%----------------------------------------------------------------------------------------
%	PACKAGES AND OTHER DOCUMENT CONFIGURATIONS
%----------------------------------------------------------------------------------------

\documentclass[twoside,twocolumn]{article}

\usepackage{blindtext} % Package to generate dummy text throughout this template 

\usepackage[sc]{mathpazo} % Use the Palatino font
\usepackage[T1]{fontenc} % Use 8-bit encoding that has 256 glyphs
\linespread{1.05} % Line spacing - Palatino needs more space between lines
\usepackage{microtype} % Slightly tweak font spacing for aesthetics

\usepackage[english]{babel} % Language hyphenation and typographical rules

\usepackage[hmarginratio=1:1,top=32mm,columnsep=20pt]{geometry} % Document margins
\usepackage[hang, small,labelfont=bf,up,textfont=it,up]{caption} % Custom captions under/above floats in tables or figures
\usepackage{booktabs} % Horizontal rules in tables

\usepackage{lettrine} % The lettrine is the first enlarged letter at the beginning of the text

\usepackage{enumitem} % Customized lists
\setlist[itemize]{noitemsep} % Make itemize lists more compact

\usepackage{abstract} % Allows abstract customization
\renewcommand{\abstractnamefont}{\normalfont\bfseries} % Set the "Abstract" text to bold
\renewcommand{\abstracttextfont}{\normalfont\small} % Set the abstract itself to small italic text

\usepackage{titlesec} % Allows customization of titles
\renewcommand\thesection{\Roman{section}} % Roman numerals for the sections
\renewcommand\thesubsection{\Roman{subsection}} % roman numerals for subsections
\titleformat{\section}[block]{\large\scshape\centering}{\thesection.}{1em}{} % Change the look of the section titles
\titleformat{\subsection}[block]{\large}{\thesubsection.}{1em}{} % Change the look of the section titles

\usepackage{fancyhdr} % Headers and footers
\pagestyle{fancy} % All pages have headers and footers
\fancyhead{} % Blank out the default header
\fancyfoot{} % Blank out the default footer
\fancyhead[C]{Running title $\bullet$ May 2016 $\bullet$ Vol. XXI, No. 1} % Custom header text
\fancyfoot[RO,LE]{\thepage} % Custom footer text

\usepackage{titling} % Customizing the title section

\usepackage{hyperref} % For hyperlinks in the PDF

%----------------------------------------------------------------------------------------
%	TITLE SECTION
%----------------------------------------------------------------------------------------

\setlength{\droptitle}{-4\baselineskip} % Move the title up

\pretitle{\begin{center}\Huge\bfseries} % Article title formatting
\posttitle{\end{center}} % Article title closing formatting
\title{ISO-S: An ISODATA-Based Initial Centroid Algorithm} % Article title
\author{%
  \textsc{
    Noah Gardner$^1$\thanks{ngardn10@students.kennesaw.edu}, 
    Sahar Yarmohamadi$^2$\thanks{yarmohamadishr@gmail.com},
    Anthony Phan$^1$\thanks{aphan5@students.kennesaw.edu},
    Chih-Cheng Hung$^1$\thanks{chung1@students.kennesaw.edu}}\\[1ex]
  \normalsize $^1$Kennesaw State University \\
  \normalsize College of Computing and Software Engineering \\
  \normalsize Marietta, GA, USA
  \and
  \normalsize $^2$Institution 2 \\ %TODO 
}
\date{} % Leave empty to omit a date

\renewcommand{\maketitlehookd}{%
\begin{abstract}
\noindent 
Iterative Self-Organizing Data Analysis Technique (ISODATA) is a clustering
algorithm based on K-Means that intends to find the correct number of clusters
through merging and splitting. Compared to K-Means, the ISODATA algorithm should
be resilient to initial cluster centers. However, ISODATA is also sensitive to
initial cluster centers which impacts the performance of the algorithm. This
work proposes using the ISODATA approach of splitting clusters to select quality
initial cluster centers. The proposed approach is evaluated on several benchmark
UCI datasets as well as the Salinas-A hyperspectral image dataset. The results
are evaluated using standard clustering evaluation parameters. \\ \\
\textit{\textbf{Keywords:} Clustering, ISODATA, Initial Centroid, Remote Sensing} \\
\end{abstract}
}

%----------------------------------------------------------------------------------------

\begin{document}

% Print the title
\maketitle

%----------------------------------------------------------------------------------------
%	ARTICLE CONTENTS
%----------------------------------------------------------------------------------------

\section{Introduction}

\lettrine[nindent=0em,lines=3]{R} emote sensing technology is used when
the data collection takes place at a remote distance from the subject matter,
such as the study of daily weather and long-term climate change, land cover
monitoring, ecosystem modeling of vegetation, and so on. Multispectral imaging
technology is one of the instruments for remote sensing used since the late
1960s. However, multispectral imaging systems use a low number of spectral bands
(three to six spectral bands in different systems). Increasing spectral
resolution in the hyperspectral images (HSI) with several hundred spectral bands
of the observational scene in a single acquisition makes it possible to acquire
detailed examination of observational scenes. The ground-based version of
hyperspectral images has gained immense interest. It has been applied to
numerous electronic imaging applications, including food inspection, forensic
science, medical surgery and diagnosis, and military applications. Because of
the difficulty of obtaining prior knowledge in remote sensing studies,
unsupervised classification and cluster analysis have great importance compared
to supervised classification. 

Unsupervised learning is a machine learning technique used when the learning
algorithm has no pre-existing labels and looks for patterns in the data set with
minimum human supervision and without external guidance. In contrast to
supervised learning, where a training set of inputs and observations are given
to the method, unsupervised learning uses only the inputs and should find an
interesting pattern in the data set. Two main approaches are used in
unsupervised learning, named principal component and cluster analysis. In the
clustering method, unlabeled and unsorted data points are grouped and segmented
according to similarities and differences to find structure in the inputs, group
it into meaningful clusters, and understand the data set better. There are
different algorithms for Clustering, such as K-Means, fuzzy C-Means, ISODATA,
and so on. In the K-Means algorithm, the goal is to minimize the sum of the
squared distances points have in the data set from its center. The problem with
the K-Means algorithm is that it requires a priori knowledge about the number of
classes in the data set, and choosing the cluster centers randomly may not
result in optimal partitioning of the data set.

One of the most popular clustering algorithms is called ISODATA clustering,
based on the K-means algorithm with some variations. The ISODATA method's
preference in comparison with the K-means algorithm is that the ISODATA
algorithm uses splitting and merging forms for clusters. The ISODATA method
allows starting from arbitrary cluster centers and obtains better partitioning
results than the K-Means algorithm, but it is not entirely resilient to initial
cluster centers. In this paper, a novel method using the ISODATA approach of
splitting clusters to select high-quality initial cluster centers is proposed.
This new method decreases the number of iterations and improves the overall
performance of the algorithm.

The paper is organized as follows: Section 2 introduces K-Means and ISODATA
clustering algorithms. The new proposed algorithm is explained in Section 3. We
present the experimental results in Section 4. Finally, Section 5 contains
concluding remarks.

%------------------------------------------------

\section{Methods}

\subsection{K-Means Algorithm}

K-Means algorithm is an unsupervised clustering algorithm and aims to partition
$N$ observations into $K$ clusters. Data points in the same cluster are very similar
to and with less variation. The user defines $K$, and these $K$ centers are placed
as far as possible for obtaining better clustering results. K-Means algorithm is
susceptible to initial starting cluster centers, and different initial values
result in a different classification. The next step is to assign each data point
to the nearest cluster center using Euclidean-based distance or other measures
based on the application-specific. After the assignment, the cluster centers are
updated by taking the average of all data points that belong to each cluster. We
continue this process until we meet the convergence criteria or the last
iteration. The algorithm is as follow:

\begin{itemize}
\item \textbf{Step 1:} Choose $K$ cluster centers from the data set.
\item \textbf{Step 2:} Assign each element of the data set to the cluster based
on the minimum distance from the cluster center.
\item \textbf{Step 3:} Re-calculate new cluster centers for each cluster after
the assignment in step 2, e.g., computing the mean of all patterns assigned to
each cluster.
\item \textbf{Step 4:} Continue steps 2 and 3 until Clustering convergence
occurs, or the number of iterations is exceeded.
\end{itemize}

For selecting the initial cluster centers in \textbf{Step 1}, there are two
standard methods:

\begin{itemize}
\item \textbf{Random Centers:} Explain here %TODO
\item \textbf{K-Means++:} Explain here %TODO
\end{itemize}

%------------------------------------------------

\subsection{ISODATA Algorithm}

ISODATA is another frequently used clustering algorithm and is similar to the
K-Means algorithm. However, in the ISODATA algorithm, the number of cluster
centers is determined dynamically, and it allows for a different number of
cluster centers. In contrast, the K-Means algorithm assumes that the number of
cluster centers is specified and fixated by the user. Some parameters such as
initial clusters' means, the minimum number of patterns in each cluster,
splitting and lumping parameters, and the number of iterations should be defined
and specified by the user. Like the K-Means algorithm, the first steps are
assignment all data points to the nearest cluster center and then recalculating
the cluster centers by replacing the average data point in each cluster. In the
next step, if a cluster does not reach the minimum number of clusters, the
user-specified, we eliminate it. After that, we do the splitting or lumping
process if their conditions are met. The cluster centers are updated based on
the newly grouped patterns. We continue this process until the convergence
occurs or the number of iterations is exceeded. The ISODATA algorithm is as
follow: 

\begin{itemize}
\item \textbf{Step 1:} Choose $K$ cluster centers from the data set.
\item \textbf{Step 2:} Assign each element of the data set to the cluster based
on the minimum distance from the cluster center.
\item \textbf{Step 3:} Re-calculate new cluster centers for each cluster after
the assignment in step 2, e.g., computing the mean of all patterns assigned to
each cluster.
\item \textbf{Step 4:} Compare the number of patterns in each cluster against
the user's number ($\theta n$)and discard small sets. 
\item \textbf{Step 5:} Do splitting process based on splitting parameter
($\theta e$) if this iteration is an odd-numbered iteration of the number of
clusters is less than or equal to ½ the desired.
\item \textbf{Step 6:} Do merging process based on merging parameter ($\theta
c$)if this iteration is even-numbered, or if the number of clusters is greater
than or equal to twice the number desired.
\item \textbf{Step 7:} Start the process again at step 2 with new cluster
centers after splitting or lumping.
\item \textbf{Step 8:} If this is not the last iteration, start the process
again at step 2. If this is the last iteration, then end the process.
\end{itemize}

%------------------------------------------------

\subsection{Proposed Method: ISO-S}
The proposed clustering algorithm's objective is to improve the result we obtain
from the ISODATA algorithm by decreasing the number of iterations needed for
convergence. Inspired by the ISODATA algorithm, this capability is achieved
using the splitting technique to update each iteration cluster centers. The main
inputs for this algorithm are the arbitrary initial number of cluster centers
and the processing parameters. First, the algorithm starts with the one cluster
center and assign the patterns to this cluster center. Second, the new algorithm
updates the cluster center by computing the mean of data points. Finally, the
algorithm employs a splitting method to form new cluster centers. The cluster
centers with the smallest number of members are discarded at the end of the
maximum number of cluster centers we desire to obtain an odd number. We continue
this process until we meet the convergence criteria: the maximum number of
clusters desired or until we reach the maximum number of iterations. The
algorithm is outlined below:

%------------------------------------------------
\section{Results}

\blindtext % Dummy text

%------------------------------------------------

\section{Conclusion}

\subsection{Subsection One}

A statement requiring citation \cite{Figueredo:2009dg}.
\blindtext % Dummy text

\subsection{Subsection Two}

\blindtext % Dummy text

%----------------------------------------------------------------------------------------
%	REFERENCE LIST
%----------------------------------------------------------------------------------------

\begin{thebibliography}{99} % Bibliography - this is intentionally simple in this template

\bibitem[Figueredo and Wolf, 2009]{Figueredo:2009dg}
Figueredo, A.~J. and Wolf, P. S.~A. (2009).
\newblock Assortative pairing and life history strategy - a cross-cultural
  study.
\newblock {\em Human Nature}, 20:317--330.
 
\end{thebibliography}

%----------------------------------------------------------------------------------------

\end{document}
